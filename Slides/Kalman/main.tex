\documentclass{beamer}

\pdfmapfile{+sansmathaccent.map}


\mode<presentation>
{
	\usetheme{Warsaw} % or try Darmstadt, Madrid, Warsaw, Rochester, CambridgeUS, ...
	\usecolortheme{seahorse} % or try seahorse, beaver, crane, wolverine, ...
	\usefonttheme{serif}  % or try serif, structurebold, ...
	\setbeamertemplate{navigation symbols}{}
	\setbeamertemplate{caption}[numbered]
} 


%%%%%%%%%%%%%%%%%%%%%%%%%%%%
% itemize settings


%%%%%%%%%%%%%%%%%%%%%%%%%%%%
% itemize settings

\definecolor{myhotpink}{RGB}{255, 80, 200}
\definecolor{mywarmpink}{RGB}{255, 60, 160}
\definecolor{mylightpink}{RGB}{255, 80, 200}
\definecolor{mypink}{RGB}{255, 30, 80}
\definecolor{mydarkpink}{RGB}{155, 25, 60}

\definecolor{mypaleblue}{RGB}{240, 240, 255}
\definecolor{mylightblue}{RGB}{120, 150, 255}
\definecolor{myblue}{RGB}{90, 90, 255}
\definecolor{mygblue}{RGB}{70, 110, 240}
\definecolor{mydarkblue}{RGB}{0, 0, 180}
\definecolor{myblackblue}{RGB}{40, 40, 120}

\definecolor{mygreen}{RGB}{0, 200, 0}
\definecolor{mygreen2}{RGB}{245, 255, 230}

\definecolor{mygray}{gray}{0.8}
\definecolor{mydarkgray}{RGB}{80, 80, 160}

\definecolor{mydarkred}{RGB}{160, 30, 30}
\definecolor{mylightred}{RGB}{255, 150, 150}
\definecolor{myred}{RGB}{200, 110, 110}
\definecolor{myblackred}{RGB}{120, 40, 40}

\definecolor{mygreen}{RGB}{0, 200, 0}
\definecolor{mygreen2}{RGB}{205, 255, 200}

\definecolor{mydarkcolor}{RGB}{60, 25, 155}
\definecolor{mylightcolor}{RGB}{130, 180, 250}

\setbeamertemplate{itemize items}[default]

\setbeamertemplate{itemize item}{\color{myblackblue}$\blacksquare$}
\setbeamertemplate{itemize subitem}{\color{mygblue}$\blacktriangleright$}
\setbeamertemplate{itemize subsubitem}{\color{mygray}$\blacksquare$}

\setbeamercolor{palette quaternary}{fg=white,bg=mydarkgray}
\setbeamercolor{titlelike}{parent=palette quaternary}

\setbeamercolor{palette quaternary2}{fg=black,bg=mypaleblue}
\setbeamercolor{frametitle}{parent=palette quaternary2}

\setbeamerfont{frametitle}{size=\Large,series=\scshape}
\setbeamerfont{framesubtitle}{size=\normalsize,series=\upshape}





%%%%%%%%%%%%%%%%%%%%%%%%%%%%
% block settings

\setbeamercolor{block title}{bg=red!30,fg=black}

\setbeamercolor*{block title example}{bg=mygreen!40!white,fg=black}

\setbeamercolor*{block body example}{fg= black, bg= mygreen2}


%%%%%%%%%%%%%%%%%%%%%%%%%%%%
% URL settings
\hypersetup{
	colorlinks=true,
	linkcolor=blue,
	filecolor=blue,      
	urlcolor=blue,
}

%%%%%%%%%%%%%%%%%%%%%%%%%%

\renewcommand{\familydefault}{\rmdefault}

\usepackage{amsmath}
\usepackage{mathtools}

\usepackage{subcaption}

\usepackage{qrcode}

\DeclareMathOperator*{\argmin}{arg\,min}
\newcommand{\bo}[1] {\mathbf{#1}}

\newcommand{\R}{\mathbb{R}} 
\newcommand{\T}{^\top}     

\newcommand{\dx}[1] {\dot{\mathbf{#1}}}
\newcommand{\ma}[4] {\begin{bmatrix}
		#1 & #2 \\ #3 & #4
\end{bmatrix}}
\newcommand{\myvec}[2] {\begin{bmatrix}
		#1 \\ #2
\end{bmatrix}}
\newcommand{\myvecT}[2] {\begin{bmatrix}
		#1 & #2
\end{bmatrix}}


\newcommand{\mydate}{Spring 2023}

\newcommand{\mygit}{\textcolor{blue}{\href{https://github.com/SergeiSa/Control-Theory-Slides-Spring-2023}{github.com/SergeiSa/Control-Theory-Slides-Spring-2023}}}

\newcommand{\myqr}{ \textcolor{black}{\qrcode[height=1.5in]{https://github.com/SergeiSa/Control-Theory-Slides-Spring-2023}}
}

\newcommand{\myqrframe}{
	\begin{frame}
		\centerline{Lecture slides are available via Github, links are on Moodle}
		\bigskip
		\centerline{You can help improve these slides at:}
		\centerline{\mygit}
		\bigskip
		\myqr
	\end{frame}
}


\newcommand{\bref}[2] {\textcolor{blue}{\href{#1}{#2}}}

%%%%%%%%%%%%%%%%%%%%%%%%%%%%
% code settings

\usepackage{listings}
\usepackage{color}
% \definecolor{mygreen}{rgb}{0,0.6,0}
% \definecolor{mygray}{rgb}{0.5,0.5,0.5}
\definecolor{mymauve}{rgb}{0.58,0,0.82}
\lstset{ 
	backgroundcolor=\color{white},   % choose the background color; you must add \usepackage{color} or \usepackage{xcolor}; should come as last argument
	basicstyle=\footnotesize,        % the size of the fonts that are used for the code
	breakatwhitespace=false,         % sets if automatic breaks should only happen at whitespace
	breaklines=true,                 % sets automatic line breaking
	captionpos=b,                    % sets the caption-position to bottom
	commentstyle=\color{mygreen},    % comment style
	deletekeywords={...},            % if you want to delete keywords from the given language
	escapeinside={\%*}{*)},          % if you want to add LaTeX within your code
	extendedchars=true,              % lets you use non-ASCII characters; for 8-bits encodings only, does not work with UTF-8
	firstnumber=0000,                % start line enumeration with line 0000
	frame=single,	                   % adds a frame around the code
	keepspaces=true,                 % keeps spaces in text, useful for keeping indentation of code (possibly needs columns=flexible)
	keywordstyle=\color{blue},       % keyword style
	language=Octave,                 % the language of the code
	morekeywords={*,...},            % if you want to add more keywords to the set
	numbers=left,                    % where to put the line-numbers; possible values are (none, left, right)
	numbersep=5pt,                   % how far the line-numbers are from the code
	numberstyle=\tiny\color{mygray}, % the style that is used for the line-numbers
	rulecolor=\color{black},         % if not set, the frame-color may be changed on line-breaks within not-black text (e.g. comments (green here))
	showspaces=false,                % show spaces everywhere adding particular underscores; it overrides 'showstringspaces'
	showstringspaces=false,          % underline spaces within strings only
	showtabs=false,                  % show tabs within strings adding particular underscores
	stepnumber=2,                    % the step between two line-numbers. If it's 1, each line will be numbered
	stringstyle=\color{mymauve},     % string literal style
	tabsize=2,	                   % sets default tabsize to 2 spaces
	title=\lstname                   % show the filename of files included with \lstinputlisting; also try caption instead of title
}


%%%%%%%%%%%%%%%%%%%%%%%%%%%%
% URL settings
\hypersetup{
	colorlinks=false,
	linkcolor=blue,
	filecolor=blue,      
	urlcolor=blue,
}

%%%%%%%%%%%%%%%%%%%%%%%%%%

%%%%%%%%%%%%%%%%%%%%%%%%%%%%
% tikz settings

\usepackage{tikz}
\tikzset{every picture/.style={line width=0.75pt}}


\title{Kalman Filter}
\subtitle{Control Theory, Lecture 10}
\author{by Sergei Savin}
\centering
\date{\mydate}



\begin{document}
\maketitle



\begin{frame}{Content}
\begin{itemize}
\item Measurement
%\item State Estimation
%\item Observer
%\item Observation and Control
%\item Separation principle
\end{itemize}
\end{frame}




\begin{frame}{Random variable, 1}
%\framesubtitle{How do we know the state?}
\begin{flushleft}

We can think of a \emph{random variable} $\bo{v}$ as a sequence of values $\bo{v}_1$, $\bo{v}_2$, $\bo{v}_3$, ... - sampled from a distribution.

\bigskip

Mean $\bar{\bo{v}}$ of a random variable $\bo{v}$ is denoted as:

\begin{equation}
	\bar{\bo{v}} = E[\bo{v}]
\end{equation}
%\begin{equation}
%	\bar{\bo{v}} = \underset{N \rightarrow \infty}{\text{lim}} \left( \frac{1}{N} \sum_{i = 1}^{N} \bo{v}_i \right)
%\end{equation}

Mean has a number of properties:

\begin{align}
	E[\bo{a}] &= \bo{a}, & \bo{a}= \text{const} \\
	E[\bo{x}+\bo{y}] &= E[\bo{x}] + E[\bo{y}] &\\
	E[\alpha \bo{x}] &= \alpha  E[\bo{x}]  & \alpha = \text{const}\\
	E[\bo{A} \bo{x}] &= \bo{A}  E[\bo{x}] & \bo{A} = \text{const}
\end{align}


\end{flushleft}
\end{frame}



\begin{frame}{Random variable, 2}
	%\framesubtitle{How do we know the state?}
	\begin{flushleft}
		
		Autocovariance $\bo{V} = \textbf{cov}(\bo{v}, \bo{v})$ of a random variable $\bo{v}$ is defined as:
		
		\begin{equation}
			\textbf{cov}(\bo{v}, \bo{v}) = E[(\bo{v} - E[\bo{v}])(\bo{v} - E[\bo{v}])\T]
		\end{equation}
	
		To simplify notation in the following sections, we define $\textbf{cov}(\bo{v}) = \textbf{cov}(\bo{v}, \bo{v})$. For zero-mean process $E[\bo{v}] = 0$ the formula simplifies:
		
		\begin{equation}
			\textbf{cov}(\bo{v}) = E[(\bo{v})(\bo{v})\T]
		\end{equation}
		
		Autocovariance has a number of properties:
		
		\begin{align}
			\textbf{cov}(\bo{a}) &= \bo{0}, & \bo{a}= \text{const} 
			\\
			\textbf{cov}(\bo{x}+\bo{a}) &= \textbf{cov}(\bo{x}), & \bo{a}= \text{const} 
			\\
			\textbf{cov}(\alpha \bo{x}) &= \alpha^2 \ \textbf{cov}(\bo{x}) &
		\end{align}
		
		
		
	\end{flushleft}
\end{frame}




\begin{frame}{Random variable, 3}
	%\framesubtitle{How do we know the state?}
	\begin{flushleft}
		
		A random variable $\bo{x}$ with Gaussian distribution can be fully described via its mean $\bar{\bo{x}}$ and covariance $\bo{X}$:
		
		\begin{equation}
			\bo{x} \sim \mathcal{N} (\bar{\bo{x}}, \bo{X})
		\end{equation}
	
		Let $\bo{x}$ be a random variable $\bo{x} \sim \mathcal{N} (\bar{\bo{x}}, \bo{X})$. 
		
		
	\end{flushleft}
\end{frame}




\begin{frame}{Mean of a linear transform}
	%\framesubtitle{How do we know the state?}
	\begin{flushleft}
		
		Given a constant matrix $\bo{M}$ we can define an affine transformation of $\bo{x}$:
		
		\begin{equation}
			\bo{y} = \bo{M}\bo{x}
		\end{equation}
		
		We can find mean of $\bo{y}$:
		%
		\begin{align}
			E[\bo{y}] = E[\bo{M}\bo{x}] \\
			E[\bo{y}] = \bo{M}E[\bo{x}] \\
			E[\bo{y}] = \bo{M}\bar{\bo{x}}
		\end{align}
		
		If $\bar{\bo{x}} = E[\bo{x}] = 0$, then $\bar{\bo{y}} = E[\bo{y}] = 0$.
		
	\end{flushleft}
\end{frame}



\begin{frame}{Autocovariance over linear transform}
	%\framesubtitle{How do we know the state?}
	\begin{flushleft}
		
		Assuming $\bar{\bo{x}} = E[\bo{x}] = 0$, we get $E[\bo{y}] = 0$; with that we can find autocovariance of $\bo{y}$:
		%
		\begin{align*}
			\textbf{cov}(\bo{y}) &= E[(\bo{y} - E[\bo{y}])(\bo{y} - E[\bo{y}])\T] =
			\\
			&=E[\bo{y}\bo{y}\T] =
			\\
			&=E[(\bo{M}\bo{x})(\bo{M}\bo{x})\T]=
			\\
			&=E[\bo{M}\bo{x}\bo{x}\T \bo{M}\T]=
			\\
			&=\bo{M}\bo{X} \bo{M}\T
		\end{align*}
		
	\end{flushleft}
\end{frame}



%\begin{frame}{Autocovariance over linear transform}
%	%\framesubtitle{How do we know the state?}
%	\begin{flushleft}
%		
%		Without this assumption, the covariance of $\bo{y}$ is a little more complicated:
%		%
%		\begin{align*}
%			\textbf{cov}(\bo{y}) = E[(\bo{y} - E[\bo{y}])(\bo{y} - E[\bo{y}])\T] =
%			\\
%			= E[\bo{y}\bo{y}\T
%			+ E[\bo{y}]E[\bo{y}]\T
%			- \bo{y}E[\bo{y}]\T
%			- E[\bo{y}]\bo{y}\T] =
%			\\
%			=
%			E[\bo{y}\bo{y}\T
%			+ \bar{\bo{y}}\bar{\bo{y}}\T
%			- \bo{y}\bar{\bo{y}}\T
%			- \bar{\bo{y}}\bo{y}]\T]=
%			\\
%			=
%			E[\bo{y}\bo{y}\T]
%			+ \bar{\bo{y}}\bar{\bo{y}}\T
%			- E[\bo{y}]\bar{\bo{y}}\T
%			- \bar{\bo{y}}E[\bo{y}]\T=
%			\\
%			=
%			E[\bo{y}\bo{y}\T]
%			+ \bar{\bo{y}}\bar{\bo{y}}\T
%			- \bar{\bo{y}}\bar{\bo{y}}\T
%			- \bar{\bo{y}}\bar{\bo{y}}\T=
%			\\
%			=
%			E[(\bo{M}\bo{x})(\bo{M}\bo{x})\T]
%			- (\bo{M}\bar{\bo{x}})(\bo{M}\bar{\bo{x}})\T
%			\\
%			=
%			E[\bo{M}\bo{x}\bo{x}\T \bo{M}\T]
%			- (\bo{M}\bar{\bo{x}}\bar{\bo{x}}\T\bo{M}\T)=
%			\\
%			=
%			\bo{M}\bo{X} \bo{M}\T
%			- (\bo{M}\bar{\bo{x}}\bar{\bo{x}}\T\bo{M}\T)=
%			\\
%			=
%			\bo{M}\bo{X} \bo{M}\T
%			- \bo{M}\bar{\bo{x}}\bar{\bo{x}}\T\bo{M}\T
%		\end{align*}
%		
%	\end{flushleft}
%\end{frame}





\begin{frame}{State estimation error - dynamics}
	%\framesubtitle{How do we know the state?}
	\begin{flushleft}
		
		Assume the DT-LTI dynamics takes the form:
		
		\begin{equation}
			\bo{x}_{i+1} = \bo{A} \bo{x}_i + \bo{B} \bo{u}_i + \bo{w}_i,
		\end{equation}
		
		where $\bo{w} \sim \mathcal{N} (0, \bo{Q})$ is \emph{process noise} - random input with Gaussian distribution. We can propose an open-loop observer:
		
		\begin{equation}
			\bhat{x}_{i+1} = \bo{A} \bhat{x}_i + \bo{B} \bo{u}_i, 
		\end{equation}
		
		where $\bhat{x}$ is state estimate. We can find estimation error $\btil{x} = \bo{x}_i - \bhat{x}_i$ dynamic:
		
		\begin{equation}
			\btil{x}_{i+1} = \bo{A} \btil{x}_i + \bo{w}_i
		\end{equation}
		
	\end{flushleft}
\end{frame}



\begin{frame}{State estimation error - mean}
	%\framesubtitle{How do we know the state?}
	\begin{flushleft}
		
		Assume you could pick your initial state estimate $\bhat{x}_0$ such that your initial state estimation error $\btil{x}_0$ behaves as a random variable sampled from a  Gaussian distribution $\btil{x}_0 \sim \mathcal{N} (0, \bo{P}_0)$.
		
		\bigskip
		
		Knowing mean $E[\btil{x}_i]$ we can compute $E[\btil{x}_{i+1}]$:
		
		\begin{equation}
			E[\btil{x}_{i+1}] = E[\bo{A} \btil{x}_i + \bo{w}_i] = 
			\bo{A} E[\btil{x}_i]
		\end{equation}		
	
		Since $E[\btil{x}_0] = 0$, we can conclude that $E[\btil{x}_i] = 0, \ \forall i$.
		
		
	\end{flushleft}
\end{frame}



\begin{frame}{State estimation error - covariance}
	%\framesubtitle{How do we know the state?}
	\begin{flushleft}
		
		Knowing autocovariance $\bo{P}_i$ we can compute $\bo{P}_{i+1}$:
		
		\begin{align*}
			\bo{P}_{i+1} &= E[\btil{x}_{i+1}\btil{x}_{i+1}\T] = 
			E[(\bo{A} \btil{x}_i + \bo{w}_i) (\bo{A} \btil{x}_i + \bo{w}_i)\T] = 
			\\
			&=
			E[\bo{A} \btil{x}_i \btil{x}_i\T \bo{A}\T +
			\bo{A} \btil{x}_i \bo{w}_i\T + 
			\bo{w}_i \btil{x}_i\T \bo{A}\T +
			\bo{w}_i \bo{w}_i\T]
		\end{align*}
		
		We can assume that random process $\bo{w}$ is uncorrelated with $\btil{x}$, meaning that $E[\btil{x}_i \bo{w}_i\T] = E[\bo{w}_i \btil{x}_i\T] = 0$:
		
		\begin{align*}
			\bo{P}_{i+1} 
			&=
			E[\bo{A} \btil{x}_i \btil{x}_i\T \bo{A}\T +
			\bo{w}_i \bo{w}_i\T] 
			= 
			\bo{A} \bo{P}_i \bo{A}\T +
			\bo{Q}
		\end{align*}
		
		
	\end{flushleft}
\end{frame}



\begin{frame}{Closed-loop observer, 1}
	%\framesubtitle{How do we know the state?}
	\begin{flushleft}
		
		Previously, we computed dynamics of mean and covariance of state estimation error for the case of open-loop observer. But, a stable observer with feedback is obviously preferable. We start by introducing a measurement model:
		
		\begin{equation}
			\bo{y}_i = \bo{H} \bo{x}_i + \bo{v}_i
		\end{equation}
		
		where $\bo{H}$ is a measurement matrix, $\bo{y}_i$ is measured output and $\bo{v}_i$ is a measurement noise sampled from a Gaussian distribution $\bo{v}_i \sim \mathcal{N} (0, \bo{R})$.
		
	\end{flushleft}
\end{frame}


\begin{frame}{Closed-loop observer, 2}
	%\framesubtitle{How do we know the state?}
	\begin{flushleft}
		
		
		We can propose the following modification to the observer:
		
		\begin{equation}
			\begin{cases}
				\bhat{x}_{i+1}^- = \bo{A} \bhat{x}_i + \bo{B} \bo{u}_i, \\
				\bhat{x}_{i+1} = \bhat{x}_{i+1}^- + \bo{L}_i (\bo{y}_i - \bo{H} \bhat{x}_{i+1}^-)
			\end{cases}
		\end{equation}
		
		where $\bhat{x}_{i+1}^-$ is an \emph{a priori} estimate. We can re-write all this in terms of state estimation error, defining $\btil{x}_{i+1}^- = \bo{x}_{i+1} - \bhat{x}_{i+1}^-$:
		
		\begin{equation}
	\begin{cases}
		\btil{x}_{i+1}^- = \bo{A} \btil{x}_i + \bo{w}_i, \\
		\btil{x}_{i+1} = \btil{x}_{i+1}^- - \bo{L}_i \bo{H} \btil{x}_{i+1}^- + \bo{v}_i
	\end{cases}
		\end{equation}		
		
	\end{flushleft}
\end{frame}



\begin{frame}{Closed-loop observer - mean dynamics}
	%\framesubtitle{How do we know the state?}
	\begin{flushleft}
		
		We can compute mean dynamics (\emph{propagation}):
		
				\begin{equation}
						\begin{cases}
								E[\btil{x}_{i+1}^-] = \bo{A} E[\btil{x}_i], \\
								E[\btil{x}_{i+1}] = (\bo{I} - \bo{L}_i \bo{H}) E[\btil{x}_{i+1}^-]
							\end{cases}
					\end{equation}		
		
		Since $E[\btil{x}_0] = 0$, then $E[\btil{x}_1^-] = 0$, and then $E[\btil{x}_1] = 0$, and the same for $E[\btil{x}_i] = 0$, $E[\btil{x}_i^-] = 0$.
		
	\end{flushleft}
\end{frame}



\begin{frame}{Closed-loop observer - covariance dynamics}
	%\framesubtitle{How do we know the state?}
	\begin{flushleft}
		
		We can compute autocovariance dynamics (propagation). Below is \emph{a priori} estimation error covariance:
		%
		\begin{align*}
				\bo{P}_{i+1}^- 
				&= E[\btil{x}_{i+1}^- (\btil{x}_{i+1}^-)\T] = \\
				&= \bo{A} \bo{P}_i \bo{A}\T +\bo{Q},
		\end{align*}		
	
		With that, we can find \emph{a posteriori} estimation error covariance:
	%
		\begin{align*}
		E[\btil{x}_{i+1} \btil{x}_{i+1}\T] 
			&=
			E[ (\bo{I} - \bo{L}_i \bo{H}) \btil{x}_{i+1}^- (\btil{x}_{i+1}^-)\T (\bo{I} - \bo{L}_i \bo{H})\T + \\
			&+(\bo{I} - \bo{L}_i \bo{H}) \btil{x}_{i+1}^- \bo{v}_i\T +
			\bo{v}_i (\btil{x}_{i+1}^-)\T (\bo{I} - \bo{L}_i \bo{H})\T +
			\bo{v}_i \bo{v}_i\T]
		\end{align*}
		
		Assuming that $\btil{x}_{i+1}^-$ and $\bo{v}_i$ are uncorrelated, we get $E[(\bo{I} - \bo{L}_i \bo{H}) \btil{x}_{i+1}^- \bo{v}_i\T] = 0$ and $E[\bo{v}_i (\btil{x}_{i+1}^-)\T (\bo{I} - \bo{L}_i \bo{H})\T = 0]$. With that we simplify:
		%
		\begin{align*}
			E[\btil{x}_{i+1} \btil{x}_{i+1}\T] 
			&=
			(\bo{I} - \bo{L}_i \bo{H}) \bo{P}_{i+1}^- (\bo{I} - \bo{L}_i \bo{H})\T +\bo{R} 
		\end{align*}
		
	\end{flushleft}
\end{frame}



\begin{frame}{Closed-loop observer - covariance dynamics}
	%\framesubtitle{How do we know the state?}
	\begin{flushleft}
		
		How do we pick $\bo{L}_i$? We can do it "the same way" as we did with LQR:
		
		\begin{equation}
			\bo{L}_i = \bo{P}_i \bo{H}\T \bo{R}^{-1}
		\end{equation}
	
		In practice, it can be better to compute $\bo{L}_i$ before we compute $\bo{P}_i$. The following allows us to compute $\bo{L}_i$ based on $\bo{P}_{i-1}$:
		
		\begin{equation}
			\bo{L}_i = \bo{P}_{i-1} \bo{H}\T (\bo{H} \bo{P}_{i-1} \bo{H}\T + \bo{R})^{-1}
		\end{equation}
	
	
	\end{flushleft}
\end{frame}



\begin{frame}{Further reading}
	%\framesubtitle{How do we know the state?}
	\begin{flushleft}
		
		\begin{itemize}
			\item Simon, D., 2006. Optimal state estimation: Kalman, H infinity, and nonlinear approaches. John Wiley \& Sons.
		\end{itemize}
		
	\end{flushleft}
\end{frame}


\myqrframe



\begin{frame}
	%\framesubtitle{How do we know the state?}
	\begin{flushleft}
		
		\centering{\Huge Appendix A}
		
	\end{flushleft}
\end{frame}

\begin{frame}{Mean of an affine transform}
	%\framesubtitle{How do we know the state?}
	\begin{flushleft}
		
		Given a constant vector $\bo{c}$ and a constant matrix $\bo{M}$ we can define an affine transformation of $\bo{x}$:
		
		\begin{equation}
			\bo{y} = \bo{M}\bo{x} + \bo{c}
		\end{equation}
		
		We can find mean of $\bo{y}$:
		%
		\begin{align}
			E[\bo{y}] = E[\bo{M}\bo{x} + \bo{c}] \\
			E[\bo{y}] = \bo{M}E[\bo{x}] + \bo{c} \\
			E[\bo{y}] = \bo{M}\bar{\bo{x}} + \bo{c}
		\end{align}
		
		
	\end{flushleft}
\end{frame}



\begin{frame}{Autocovariance with zero mean}
	%\framesubtitle{How do we know the state?}
	\begin{flushleft}
		
		Assuming $E[\bo{x}] = 0$, we can find covariance of $\bo{y}$:
		%
		\begin{align*}
			\textbf{cov}(\bo{y}) = E[(\bo{y} - E[\bo{y}])(\bo{y} - E[\bo{y}])\T] =
			\\
			= E[\bo{y}\bo{y}\T
			+ E[\bo{y}]E[\bo{y}]\T
			- \bo{y}E[\bo{y}]\T
			- E[\bo{y}]\bo{y}\T] =
			\\
			=
			E[\bo{y}\bo{y}\T
			+ \bar{\bo{y}}\bar{\bo{y}}\T
			- \bo{y}\bar{\bo{y}}\T
			- \bar{\bo{y}}\bo{y}]\T]=
			\\
			=
			E[\bo{y}\bo{y}\T]
			+ \bar{\bo{y}}\bar{\bo{y}}\T
			- E[\bo{y}]\bar{\bo{y}}\T
			- \bar{\bo{y}}E[\bo{y}]\T=
			\\
			=
			E[\bo{y}\bo{y}\T]
			+ \bar{\bo{y}}\bar{\bo{y}}\T
			- \bar{\bo{y}}\bar{\bo{y}}\T
			- \bar{\bo{y}}\bar{\bo{y}}\T
			\\
			=
			E[(\bo{M}\bo{x} + \bo{c})(\bo{M}\bo{x} + \bo{c})\T]
			- \bo{c}\bo{c}\T
			\\
			=
			E[\bo{M}\bo{x}\bo{x}\T \bo{M}\T+
			\bo{c}\bo{c}\T+
			\bo{M}\bo{x}\bo{c}\T+
			\bo{c}\bo{x}\T \bo{M}\T]
			- \bo{c}\bo{c}\T=
			\\
			=
			\bo{M}\bo{X} \bo{M}\T+
			\bo{M}\bar{\bo{x}}\bo{c}\T+
			\bo{c}\bar{\bo{x}}\T \bo{M}\T=
			\\
			=
			\bo{M}\bo{X} \bo{M}\T
		\end{align*}
		
	\end{flushleft}
\end{frame}

\begin{frame}{Autocovariance over affine transform}
	%\framesubtitle{How do we know the state?}
	\begin{flushleft}
		
		Without this assumption, the covariance of $\bo{y}$ is a little more complicated:
		%
		\begin{align*}
			\textbf{cov}(\bo{y}) = E[(\bo{y} - E[\bo{y}])(\bo{y} - E[\bo{y}])\T] =
			\\
			= E[\bo{y}\bo{y}\T
			+ E[\bo{y}]E[\bo{y}]\T
			- \bo{y}E[\bo{y}]\T
			- E[\bo{y}]\bo{y}\T] =
			\\
			=
			E[\bo{y}\bo{y}\T
			+ \bar{\bo{y}}\bar{\bo{y}}\T
			- \bo{y}\bar{\bo{y}}\T
			- \bar{\bo{y}}\bo{y}]\T]=
			\\
			=
			E[\bo{y}\bo{y}\T]
			+ \bar{\bo{y}}\bar{\bo{y}}\T
			- E[\bo{y}]\bar{\bo{y}}\T
			- \bar{\bo{y}}E[\bo{y}]\T=
			\\
			=
			E[\bo{y}\bo{y}\T]
			+ \bar{\bo{y}}\bar{\bo{y}}\T
			- \bar{\bo{y}}\bar{\bo{y}}\T
			- \bar{\bo{y}}\bar{\bo{y}}\T
			\\
			=
			E[(\bo{M}\bo{x} + \bo{c})(\bo{M}\bo{x} + \bo{c})\T]
			- (\bo{M}\bar{\bo{x}} + \bo{c})(\bo{M}\bar{\bo{x}} + \bo{c})\T
			\\
			=
			E[\bo{M}\bo{x}\bo{x}\T \bo{M}\T+
			\bo{c}\bo{c}\T+
			\bo{M}\bo{x}\bo{c}\T+
			\bo{c}\bo{x}\T \bo{M}\T]
			-\\
			- (\bo{M}\bar{\bo{x}}\bar{\bo{x}}\T\bo{M}\T +
			\bo{M}\bar{\bo{x}}\bo{c}\T+
			\bo{c}\bar{\bo{x}}\T\bo{M}\T+
			\bo{c}\bo{c}\T)=
			\\
			=
			\bo{M}\bo{X} \bo{M}\T+
			\bo{c}\bo{c}\T+
			\bo{M}\bar{\bo{x}}\bo{c}\T+
			\bo{c}\bar{\bo{x}}\T \bo{M}\T
			-\\
			- (\bo{M}\bar{\bo{x}}\bar{\bo{x}}\T\bo{M}\T +
			\bo{M}\bar{\bo{x}}\bo{c}\T+
			\bo{c}\bar{\bo{x}}\T\bo{M}\T+
			\bo{c}\bo{c}\T)=
			\\
			=
			\bo{M}\bo{X} \bo{M}\T
			- \bo{M}\bar{\bo{x}}\bar{\bo{x}}\T\bo{M}\T
		\end{align*}
		
	\end{flushleft}
\end{frame}




\end{document}
